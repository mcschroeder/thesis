\section{Semantics of STM}
\label{sec:stm-orig-semantics}

I now formalize my design by giving an operational semantics.
In this section, I present an overview of the original semantics of STM Haskell as described by \textcite{harris-et-al-2005}, including the revised exception semantics from their post-publication appendix.
Using this as a foundation, I will then add the changes necessary to implement finalizers in \Cref{sec:stm-fin-semantics}.
There I will also include the invariant semantics described by \textcite{harris-peytonjones-2006}.

Throughout my presentation of these semantics, I aimed to keep as close to the original papers as possible to ease cross-referencing.
However, a few minor modifications have been made to better reflect the current state of GHC, such as changing the names of some of the primitives.

\bigskip
\Cref{fig:orig-syntax,fig:orig-admin,fig:orig-io,fig:orig-stm} give the complete syntax and semantics of the original version of STM Haskell.
% key idea, execution, I/O concurrency, STM atomicity
The key idea is to separate \emph{I/O transitions} (``$\rightarrow$'', \Cref{fig:orig-io}) from \emph{STM transitions} (``$\Rightarrow$'', \Cref{fig:orig-stm}).
Execution proceeds by non-deterministically choosing a thread and performing a single I/O transition.
Thus, the execution of different threads may be interleaved and we have concurrency on the I/O level.
However, STM transitions can not interleave.
They are only ever performed as premises of the \term{atomically} operator and the resulting state change is thus regarded as a single atomic step on the I/O level.

% logs and rollback
Simply requiring that \term{atomically} must reduce (on the STM level) to either \term{return} or \term{throw}, and not \term{retry}, obviates the need for modeling transaction logs or rollback.
These matters are left for the implementation.

\clearpage

\paragraph{Syntax}

% syntax of values and terms
The syntax of values and terms (\Cref{fig:orig-syntax}) is entirely conventional, except that some monadic combinators are treated as values.
%
% do notation
The \term{do}-notation used in the preceding sections is of course just the usual syntactic sugar for the monadic operators \term{>>=} and \term{return}:
\begin{align*}
\term{do \string{ x <- e; Q \string}} & \quad\equiv\quad \term{e >>= (\string\x \ -> do \string{ Q \string})}\\
\term{do \string{ e; Q \string}} & \quad\equiv\quad \term{e >>= (\string\_ -> do \string{ Q \string})}\\
\term{do \string{ e \string}} & \quad\equiv\quad \term{e}
\end{align*}

% syntax for program state
A \emph{program state} $P;\Theta$ consists of a \emph{thread soup} $P$ and a \emph{heap} $\Theta$.
The thread soup $P$ is a multi-set of threads with each thread consisting of a single term $M_t$, where $t$ is the thread ID.
The heap $\Theta$ maps references to terms.

% allocation effects
Additionally, we sometimes want to explicitly track a set of \emph{allocation effects} $\Delta$.
This is a redundant subset of the heap.
We mainly need it during exception handling, where we want to roll back heap effects but have to retain allocation effects.

% evaluation context
When describing a transition between program states, we use an \emph{evaluation context} to identify the active site of the transition.
The program evaluation context $\mathbb{P}$ arbitrarily chooses a thread from the soup.
This corresponds to the scheduler of a real implementation.
STM terms are then evaluated using the context $\mathbb{S}$.

\paragraph{Administrative transitions}

% admin transitions
A few fundamental transitions are used on both the I/O and the STM level (\Cref{fig:orig-admin}).
The rule \rul{BIND}, which implements sequential composition in the monad, as well as \rul{THROW} and \rul{RETRY} are quite ordinary.
\rul{EVAL} allows evaluation of terms.
It uses a function $\mathcal{V}$, which is entirely standard and whose definition is omitted.

% lifting
The I/O level rule \rul{ADMIN} lifts administrative rules into the I/O world. \rul{AADMIN} does the same for STM.

\begin{figure}[p]
\begin{adjustbox}{frame=0.5pt 1em,max width=\textwidth,center,tabular=@{}r >{$}r<{$} >{$}c<{$} >{$}l<{$}@{}}
& x,y & \in & Variable \\
& r,t & \in & Name \\
& c & \in & Char \\
\\
Value & V &  ::=  & r \svert c \svert \term{\textbackslash}\ x \term{->}\ M\\
& & \vert & \term{return}\ M \svert M\ \term{>>=}\ N \\
& & \vert & \term{putChar}\ c \svert \term{getChar} \\
& & \vert & \term{throw}\ M \svert \term{catch}\ M\ N \\
& & \vert & \term{retry} \svert M\ \term{`orElse`}\ N \\
& & \vert & \term{newTVar} \\
& & \vert & \term{readTVar}\ r \svert \term{writeTVar}\ r\ M \\
& & \vert & \term{forkIO}\ M \\
& & \vert & \term{atomically}\ M \\
\\
Term & M,N & ::= & x \svert V \svert M\ N \svert \ldots \\
\\
Thread soup & P,Q & ::= & M_t \svert (P\ \vert\ Q) \\
Heap & \Theta & ::= & r \hookrightarrow M \\
Allocations & \Delta & ::= & r \hookrightarrow M \\
\\
Evaluation contexts & \mathbb{S} & ::= & [\cdot] \svert \mathbb{S}\ \term{>>=}\ M \\
& \mathbb{P} & ::= & \mathbb{S}_t \svert (\mathbb{P}\ \vert\ P) \svert (P\ \vert\ \mathbb{P}) \\
Action & a & ::= & !c \svert ?c \svert \epsilon
\end{adjustbox}
\caption{The original syntax of STM Haskell \parencite{harris-et-al-2005}}
\label{fig:orig-syntax}
\end{figure}

\begin{figure}[p]
\begin{adjustbox}{frame=0.5pt 1em,max width=\textwidth,center,tabular=@{}c@{}}
\fbox{Administrative transitions \qquad $M\ \rightarrow\ N$}\\
\\\\
\begin{tabular}{@{}>{$}r<{$} >{$}c<{$} >{$}l<{$} l@{}}
M & \rightarrow & V \quad \text{if }\mathcal{V}[\![M]\!] = V \text{ and } M \not\equiv V & \rul{EVAL}
\end{tabular}\\
\\
\begin{tabular}{@{}>{$}r<{$} >{$}c<{$} >{$}l<{$} l@{}}
\term{return}\ N\ \term{>>=}\ M & \rightarrow & M\ N & \rul{BIND}\\
\term{throw}\ N\ \term{>>=}\ M & \rightarrow & \term{throw}\ N & \rul{THROW}\\
\term{retry}\ \term{>>=}\ M & \rightarrow & \term{retry} & \rul{RETRY}\\
\end{tabular}
\end{adjustbox}
\caption{Administrative transitions of STM Haskell \parencite{harris-et-al-2005}}
\label{fig:orig-admin}
\end{figure}


\clearpage

\paragraph{I/O transitions}

% io transitions
A top level I/O transition (\Cref{fig:orig-io}) is of the form
$$P;\Theta \enspace\xrightarrow{a}\enspace Q;\Theta'.$$ 
It takes a program state $P;\Theta$ to a new state $Q;\Theta'$ while performing the input/output action $a$.
The actions $!c$ and $?c$ denote writing a character to the standard output and reading a character from the standard input, respectively.
The silent action $\epsilon$ does nothing and is usually omitted.
These are the only actions in our model, a real system would of course have many more.

% rules
The first two rules, \rul{PUTC} and \rul{GETC}, are applicable when the active term is \term{putChar} or \term{getChar}, respectively.
The transition carries out the appropriate action and replaces the term with a \term{return} containing the result.
The rule \rul{FORK} creates a new thread, with a freshly chosen thread ID $t$, and adds it to the thread soup.
\rul{CATCH1} and \rul{CATCH2} handle I/O level exceptions in the standard way.

% atomically rules
More interesting are the rules \rul{ARET} and \rul{ATHROW}.
They define the semantics of atomic blocks.
We can see that the only way to perform zero or more STM transitions (``$\Rightarrow$'') is by performing a single I/O transition (``$\rightarrow$''), brought about by the \term{atomically} keyword.
Interleaving of STM transitions is therefore not possible.
Also, these are the only I/O level rules that affect the heap, which means that the heap can only be mutated inside an atomic block.

% aret
The rule \rul{ARET} concerns STM terms that ultimately evaluate to \term{return} statements.
This means the STM transaction was successfully completed and the updated heap $\Theta'$ becomes visible on the I/O level.
%
% athrow
If, on the other hand, the STM term evaluates to $\term{throw}\ N$, i.e.\ if the STM transaction throws an exception $N$, the exception is propagated to the I/O level and the new heap discarded.
Only the allocation effects $\Delta'$ are kept and made globally visible.
This is necessary because the exception value $N$ may contain references to variables that were allocated inside the transaction and we do not want to leave dangling pointers around.


\begin{figure}[p]
\begin{adjustbox}{frame=0.5pt 1em,max width=\textwidth,center,tabular=@{}c@{}}
\fbox{I/O transitions \qquad$P;\ \Theta\ \xrightarrow{a}\ Q;\ \Theta'$}\\
\\
\begin{adjustbox}{max width=\linewidth,tabular=@{}>{$}r<{$} >{$}c<{$} >{$}l<{$} l}
\ctxP{\term{putChar}\ c};\ \Theta & \xrightarrow{!c} & \ctxP{\term{return ()}};\ \Theta & \rul{PUTC} \\
\ctxP{\texttt{getChar}};\ \Theta & \xrightarrow{?c} & \ctxP{\term{return}\ c};\ \Theta & \rul{GETC} \\
\ctxP{\texttt{forkIO } M};\ \Theta & \rightarrow & (\ctxP{\term{return}\ t}\ \vert\ M_t);\ \Theta \quad t \notin \mathbb{P},\Theta,M & \rul{FORK} \\
\ctxP{\term{catch}\ (\term{return}\ M)\ N};\ \Theta & \rightarrow & \ctxP{\term{return}\ M};\ \Theta & \rul{CATCH1} \\
\ctxP{\term{catch}\ (\term{throw}\ P)\ N};\ \Theta & \rightarrow & \ctxP{N\ P};\ \Theta & \rul{CATCH2} \\
\end{adjustbox}\\
\\
\inference{M\ \rightarrow\ N}{\ctxP{M};\ \Theta\ \rightarrow\ \ctxP{N};\ \Theta}[\rul{ADMIN}]\\
\\
\inference{M;\ \Theta,\{\}\ \xRightarrow{*}\ \term{return}\ N;\ \Theta',\Delta'}{\ctxP{\term{atomically}\ M};\ \Theta\ \rightarrow\ \ctxP{\term{return}\ N};\ \Theta'}[\rul{ARET}]\\
\\
\inference{M;\ \Theta,\{\}\ \xRightarrow{*}\ \term{throw}\ N;\ \Theta',\Delta'}{\ctxP{\term{atomically}\ M};\ \Theta\ \rightarrow\ \ctxP{\term{throw}\ N};\ \Theta \cup \Delta'}[\rul{ATHROW}]
\end{adjustbox}
\caption{I/O transitions of STM Haskell \parencite{harris-et-al-2005}}
\label{fig:orig-io}
\end{figure}

\clearpage

\paragraph{STM transitions}

% stm transitions
An \emph{STM transition} (\Cref{fig:orig-stm}) is of the form 
$$M;\Theta,\Delta \enspace\Rightarrow\enspace N;\Theta',\Delta'.$$
$\Theta$ is again the heap, while $\Delta$ redundantly records the allocation effects.
Note that none of these effects are visible globally until we make a transition on the I/O level that exposes them.
All STM transitions stay completely within the STM layer, except for rule \rul{AADMIN}, which just lifts pure administrative transitions.

% read, write, new
Most of the STM rules are quite standard.
\rul{READ}, \rul{WRITE} and \rul{NEW} describe the reading, writing and allocating of mutable variables.
Note how \rul{NEW} makes allocation effects explicit by modifying $\Delta$.

% or1-3
The rules \rul{OR1-3} handle the kind of nested transactions made possible by the \term{orElse} combinator.\footnote{Not to be confused with the nesting of \term{atomically} calls described in \Cref{subsec:stm-example-nesting}.}
When encountering a term of the form $M_1\ \term{orElse}\ M_2$ we always begin by inspecting the left branch $M_1$.
If this evaluates to a \term{return} \rul{OR1} or a \term{throw} \rul{OR2}, the result is propagated and all memory effects are retained.
Keep in mind that we are still on the STM level.
If the whole STM transaction finally evaluates to \term{throw}, then its memory effects are of course globally discarded by the I/O level rule \rul{ATHROW}.
If we encounter a \term{retry} when evaluating $M_1$, rule \rul{OR3} applies, discarding all memory effects and continuing with the evaluation of the right branch $M_2$.

% xstm1-3
Rules \rul{XSTM1-3} describe exception handling within STM by use of the \term{catch} combinator.
When evaluating the body $M$ of a term $\term{catch}\ M\ N$, we start with the current heap $\Theta$ and an empty set of allocation effects.
On successful evaluation \rul{XSTM1}, all effects are preserved and the catch handler $N$ is ignored.
If $M$ throws an exception \rul{XSTM2}, that exception is given to the handler $N$ and only the allocation effects of $M$ are preserved.
If $M$ evaluates to \term{retry} \rul{XSTM3}, all effects are discarded.

This last set of rules was introduced in the post-publication appendix of \textcite{harris-et-al-2005} to alleviate a subtle issue concerning leaking of effects.
For a more elaborate discussion of this issue and of the whole system, refer to that paper.

\begin{figure}[p]
\begin{adjustbox}{frame=0.5pt 1em,max width=\textwidth,center,tabular=@{}c@{}}
\fbox{STM transitions \qquad $M;\ \Theta,\Delta\ \Rightarrow\ N;\ \Theta',\Delta'$}\\
\\
\begin{adjustbox}{max width=\linewidth,tabular=@{}>{$}r<{$} >{$}c<{$} >{$}l<{$} >{$}l<{$} l@{}}
\ctxS{\term{readTVar}\ r};\ \Theta,\Delta & \Rightarrow & \ctxS{\term{return}\ \Theta(r)};\ \Theta,\Delta & \text{if } r \in dom(\Theta) & \rul{READ} \\
\ctxS{\term{writeTVar}\ r\ M};\ \Theta,\Delta & \Rightarrow & \ctxS{\term{return} \texttt{()}};\ \Theta[r\mapsto M],\Delta & \text{if } r \in dom(\Theta) & \rul{WRITE} \\
\ctxS{\term{newTVar}\ M};\ \Theta,\Delta & \Rightarrow & \ctxS{\term{return}\ r};\ \Theta[r\mapsto M],\Delta[r\mapsto M] & r \notin dom(\Theta) & \rul{NEW}
\end{adjustbox}\\
\\
\inference{M \quad \rightarrow \quad N}{\ctxS{M};\ \Theta,\Delta\  \Rightarrow\ \ctxS{N};\ \Theta,\Delta}[\rul{AADMIN}]\\
\\
\inference{M_1;\ \Theta,\Delta\ \xRightarrow{*}\ \term{return}\ N;\ \Theta', \Delta'}{\ctxS{M_1\ \term{`orElse`}\ M_2};\ \Theta,\Delta\ \Rightarrow\  \ctxS{\term{return}\ N};\ \Theta',\Delta'}[\rul{OR1}]\\
\\
\inference{M_1;\ \Theta,\Delta\ \xRightarrow{*}\ \term{throw}\ N;\ \Theta',\Delta'}{\ctxS{M_1\ \term{`orElse`}\ M_2};\ \Theta,\Delta\ \Rightarrow\ \ctxS{\term{throw}\ N};\ \Theta', \Delta'}[\rul{OR2}]\\
\\
%TODO why are the allocation effects not preserved?
\inference{M_1;\ \Theta,\Delta\ \xRightarrow{*}\ \term{retry} ;\ \Theta',\Delta'}{\ctxS{M_1\ \term{`orElse`}\ M_2};\ \Theta,\Delta\ \Rightarrow\ \ctxS{M_2};\ \Theta,\Delta}[\rul{OR3}]\\
\\
%TODO ghc discards all invariants added as part of a catch block, contrary to these semantics. this is most probably a bug (it does not discard invariants added in the exception handler of catch) the question is: should invariants added before a throw in a catch block also be retained?
\inference{M;\ \Theta,\{\}\ \xRightarrow{*}\ \term{return}\ P;\ \Theta',\Delta'}{\ctxS{\term{catch}\ M\ N};\ \Theta,\Delta\ \Rightarrow\ \ctxS{\term{return}\ P};\ \Theta',(\Delta\cup\Delta')}[\rul{XSTM1}]\\
\\
\inference{M;\ \Theta,\{\}\ \xRightarrow{*}\ \term{throw}\ P;\ \Theta',\Delta'}{\ctxS{\term{catch}\ M\ N};\ \Theta,\Delta\ \Rightarrow\ \ctxS{N\ P};\ (\Theta\cup\Delta'),(\Delta\cup\Delta')}[\rul{XSTM2}]\\
\\
%TODO again, why are the allocation effects not preserved?
\inference{M;\ \Theta,\{\}\ \xRightarrow{*}\ \term{retry};\ \Theta',\Delta'}{\ctxS{\term{catch}\ M\ N};\ \Theta,\Delta\ \Rightarrow\ \ctxS{\term{retry}};\ \Theta,\Delta}[\rul{XSTM3}]
\end{adjustbox}
\caption{STM transitions of STM Haskell \parencite{harris-et-al-2005}}
\label{fig:orig-stm}
\end{figure}

\clearpage

%------------------------------------------------------------

\section{Semantics of finalizers}
\label{sec:stm-fin-semantics}

Now that we are familiar with the original STM semantics, we can extend them to support finalizers.
First, let us realize that \term{atomically} is trivially expressed in terms of \term{atomicallyWithIO}:
\begin{align*}
\term{atomically m} & \quad\equiv\quad \term{atomicallyWithIO m return}
\end{align*}
We can therefore use \term{atomicallyWithIO} to replace the old \term{atomically} completely.
This keeps things simple and saves us from an annoying duplication of rules.
\Cref{fig:step1-io} gives the revised I/O transitions.
There is no need to change anything on the STM level.

% athrow1
Rule \rul{ATHROW} has been replaced by \rul{ATHROW1}, which is identical except that it now uses \term{atomicallyWithIO}.
Like before, if the STM term evaluates to an exception, the changed heap $\Theta'$ is discarded and only the allocation effects $\Delta'$ are kept.
The finalizer $F$ given to \term{atomicallyWithIO} is simply ignored in this case.

% aret
The heart of the change is to be found in rule \rul{ARET}, which now has an additional premise:
after the atomic block returns, the finalizer given to \term{atomicallyWithIO} must also evaluate to \term{return}.
It does so by making zero or more \emph{I/O-level} transitions.
Note that these transitions begin with the original unchanged heap $\Theta$.
The finalizer executes in a global context, where the results of the STM transaction are not yet visible.
Thus the finalizer cannot know about the modified heap $\Theta'$.
(It must know about the allocation effects $\Delta'$, however:
$N$ could contain a reference to a newly allocated variable and we do not want any dangling pointers.)
The final heap resulting from the \rul{ARET} transition is the union of the STM result heap $\Theta'$ with the result heap of the I/O transition, $\hat{\Theta}$.

% athrow2
The new rule \rul{ATHROW2} handles exceptions that occur during the coupled I/O transition.
Like with \rul{ATHROW1}, the STM result heap $\Theta'$ is discarded.
However, the I/O result heap $\hat{\Theta}$ \emph{must be kept}, because it could have already been seen by other I/O level transitions.
This is what we mean when we speak of irrevocable I/O.

\begin{figure}[!h]
\begin{adjustbox}{frame=0.5pt 1em,max width=\textwidth,center,tabular=@{}c@{}}
\fbox{I/O transitions \qquad$P;\ \Theta\ \xrightarrow{a}\ Q;\ \Theta'$}\\
\\
\begin{adjustbox}{max width=\linewidth,tabular=@{}>{$}r<{$} >{$}c<{$} >{$}l<{$} l@{}}
\ctxP{\term{putChar}\ c};\ \Theta & \xrightarrow{!c} & \ctxP{\term{return ()}};\ \Theta & \rul{PUTC} \\
\ctxP{\term{getChar}};\ \Theta & \xrightarrow{?c} & \ctxP{\term{return}\ c};\ \Theta & \rul{GETC} \\
\ctxP{\term{forkIO } M};\ \Theta & \rightarrow & (\ctxP{\term{return}\ t}\ \vert\ M_t);\ \Theta \quad t \notin \mathbb{P},\Theta,M & \rul{FORK} \\
\ctxP{\term{catch}\ (\term{return}\ M)\ N};\ \Theta & \rightarrow & \ctxP{\term{return}\ M};\ \Theta & \rul{CATCH1} \\
\ctxP{\term{catch}\ (\term{throw}\ P)\ N};\ \Theta & \rightarrow & \ctxP{N\ P};\ \Theta & \rul{CATCH2} \\
\end{adjustbox}\\
\\
\inference{M\ \rightarrow\ N}{\ctxP{M};\ \Theta\ \rightarrow\ \ctxP{N};\ \Theta}[\rul{ADMIN}]\\
\\
\inference{M;\ \Theta,\{\}\ \xRightarrow{*}\ \term{return}\ N;\ \Theta',\Delta'\\ \colorA{F\ N;\ (\Theta \cup \Delta')\ \xrightarrow{*}\ \term{return}\ P;\ \hat{\Theta}}}{\ctxP{\colorA{\term{atomicallyWithIO}}\ M\ \colorA{F}};\ \Theta\ \rightarrow\ \ctxP{\term{return}\ \colorA{P}};\ \Theta'\colorA{{}\cup{}\hat{\Theta}}}[\rul{ARET}]\\
\\
\inference{M;\ \Theta,\{\}\ \xRightarrow{*}\ \term{throw}\ N;\ \Theta',\Delta'}{\ctxP{\colorA{\term{atomicallyWithIO}}\ M\ \colorA{F}};\ \Theta\ \rightarrow\ \ctxP{\term{throw}\ N};\ \Theta \cup \Delta'}[\rul{ATHROW\colorA{1}}]\\
\\
\colorA{\inference{M;\ \Theta,\{\}\ \xRightarrow{*}\ \term{return}\ N;\ \Theta',\Delta'\\ F\ N;\ (\Theta \cup \Delta')\ \xrightarrow{*}\ \term{throw}\ P;\ \hat{\Theta}}{\ctxP{\term{atomicallyWithIO}\ M\ F};\ \Theta\ \rightarrow\ \ctxP{\term{return}\ P};\ \hat{\Theta}}[\rul{ATHROW2}]}
\end{adjustbox}
\caption{I/O transitions of STM Haskell with \colorA{finalizers}}
\label{fig:step1-io}
\end{figure}

\clearpage

%------------------------------------------------------------

\subsection{Nesting}

The attentive reader might have noticed that in the extended semantics given above the following term is perfectly valid:
$$
\term{atomicallyWithIO}\ M_1\ (\term{atomicallyWithIO}\ M_2\ F)
$$

As I have discussed in \Cref{sec:stm-overview}, this kind of nesting is actually pretty useful in practice.
However, it does open up the possibility of deadlocks if the term $M_2$ writes to any of the memory locations referenced by $M_1$.
Such behavior should be explicitly prohibited in the semantics, which need to be extended a bit further to do so (\Cref{fig:step2-stm,fig:step2-io}).

% sigma
Firstly, both I/O and STM transitions will carry around a new piece of state $\Sigma$, which is a set of labeled memory references:
$$
\Sigma ::= r \times (\textsc{r}\ \vert\ \textsc{w})
$$
It is a record of all memory operations done by a transaction.
The labels \textsc{r} and \textsc{w} denote whether a particular memory location was read from or written to, respectively.
As a shorthand notation, we define $\Sigma_\textsc{w}=\{r \mid (r,\textsc{w}) \in \Sigma\}$, i.e.\ the set of references in $\Sigma$ that have been used in a write operation.
$\Sigma_\textsc{r}$ and $\Sigma_\textsc{rw}$ are defined analogously.
We also define $\Sigma\vert_\Delta=\{ (r,\textsc{w}) \in \Sigma \mid r \in dom(\Delta)\}$, which restricts $\Sigma$ to those labelled references that correspond to the allocations recorded in $\Delta$.

% stm
Secondly, the STM transition rules \rul{READ}, \rul{WRITE} and \rul{NEW} modify $\Sigma$ by recording their respective operations.
All other rules treat $\Sigma$ in very much the same way as the heap.

% io
Finally, the purpose of $\Sigma$ becomes clear on the I/O-level, where rules \rul{ARET} and \rul{ATHROW2} include a new condition above the line:
the set of memory locations read from or written to during the STM transition ($\Sigma'_\textsc{rw}$) and the set of memory locations written to during the finalizing I/O transition ($\hat{\Sigma}_\textsc{w}$) must be entirely distinct.
There must not be a single shared reference between the two phases -- except for references that the finalizer has only read.
This is exactly the restriction necessary to avoid deadlocks.

\begin{figure}[p]
\begin{adjustbox}{frame=0.5pt 1em,max width=\textwidth,center,tabular=@{}c@{}}
\fbox{STM transitions \qquad $M;\ \Theta,\Delta\colorB{,\Sigma}\ \Rightarrow\ N;\ \Theta',\Delta'\colorB{,\Sigma'}$}\\
\\
\begin{adjustbox}{max width=\linewidth,tabular=@{}>{$}r<{$} >{$}c<{$} >{$}l<{$} >{$}l<{$} l@{}}
\ctxS{\term{readTVar}\ r};\ \Theta,\Delta\colorB{,\Sigma} & \Rightarrow & \ctxS{\term{return}\ \Theta(r)};\ \Theta,\Delta\colorB{,(\Sigma\cup\{(r,\textsc{r})\})} & \text{if } r \in dom(\Theta) & \rul{READ} \\
\ctxS{\term{writeTVar}\ r\ M};\ \Theta,\Delta\colorB{,\Sigma} & \Rightarrow & \ctxS{\term{return ()}};\ \Theta[r\mapsto M],\Delta\colorB{,(\Sigma\cup\{(r,\textsc{w})\})} & \text{if } r \in dom(\Theta) & \rul{WRITE} \\
\ctxS{\term{newTVar}\ M};\ \Theta,\Delta\colorB{,\Sigma} & \Rightarrow & \ctxS{\term{return}\ r};\ \Theta[r\mapsto M],\Delta[r\mapsto M]\colorB{,(\Sigma\cup\{(r,\textsc{w})\})} & r \notin dom(\Theta) & \rul{NEW}
\end{adjustbox}\\
\\
\inference{M \quad \rightarrow \quad N}{\ctxS{M};\ \Theta,\Delta\colorB{,\Sigma}\  \Rightarrow\ \ctxS{N};\ \Theta,\Delta\colorB{,\Sigma}}[\rul{AADMIN}]\\
\\
\inference{M_1;\ \Theta,\Delta\colorB{,\Sigma}\ \xRightarrow{*}\ \term{return}\ N;\ \Theta', \Delta'\colorB{,\Sigma'}}{\ctxS{M_1\ \term{`orElse`}\ M_2};\ \Theta,\Delta\colorB{,\Sigma}\ \Rightarrow\  \ctxS{\term{return}\ N};\ \Theta',\Delta'\colorB{,\Sigma'}}[\rul{OR1}]\\
\\
\inference{M_1;\ \Theta,\Delta\colorB{,\Sigma}\ \xRightarrow{*}\ \term{throw}\ N;\ \Theta',\Delta'\colorB{,\Sigma'}}{\ctxS{M_1\ \term{orElse}\ M_2};\ \Theta,\Delta\colorB{,\Sigma}\ \Rightarrow\ \ctxS{\term{throw}\ N};\ \Theta',\Delta'\colorB{,\Sigma'}}[\rul{OR2}]\\
\\
\inference{M_1;\ \Theta,\Delta\colorB{,\Sigma}\ \xRightarrow{*}\ \term{retry} ;\ \Theta',\Delta'\colorB{,\Sigma'}}{\ctxS{M_1\ \term{orElse}\ M_2};\ \Theta,\Delta\colorB{,\Sigma}\ \Rightarrow\ \ctxS{M_2};\ \Theta,\Delta\colorB{,\Sigma}}[\rul{OR3}]\\
\\
\inference{M;\ \Theta,\{\}\colorB{,\Sigma}\ \xRightarrow{*}\ \term{return}\ P;\ \Theta',\Delta'\colorB{,\Sigma'}}{\ctxS{\term{catch}\ M\ N};\ \Theta,\Delta\colorB{,\Sigma}\ \Rightarrow\ \ctxS{\term{return}\ P};\ \Theta',(\Delta\cup\Delta')\colorB{,\Sigma'}}[\rul{XSTM1}]\\
\\
\begin{adjustbox}{max width=\linewidth,tabular=@{}c@{}}
\inference{M;\ \Theta,\{\}\colorB{,\Sigma}\ \xRightarrow{*}\ \term{throw}\ P;\ \Theta',\Delta'\colorB{,\Sigma'}}{\ctxS{\term{catch}\ M\ N};\ \Theta,\Delta\colorB{,\Sigma}\ \Rightarrow\ \ctxS{N\ P};\ (\Theta\cup\Delta'),(\Delta\cup\Delta')\colorB{,(\Sigma\cup\Sigma'\vert_{\Delta'})}}[\rul{XSTM2}]
\end{adjustbox}\\
\\
\inference{M;\ \Theta,\{\}\colorB{,\Sigma}\ \xRightarrow{*}\ \term{retry};\ \Theta',\Delta'\colorB{,\Sigma'}}{\ctxS{\term{catch}\ M\ N};\ \Theta,\Delta\colorB{,\Sigma}\ \Rightarrow\ \ctxS{\term{retry}};\ \Theta,\Delta\colorB{,\Sigma}}[\rul{XSTM3}]
\end{adjustbox}
\caption{STM transitions of STM Haskell with \colorA{finalizers} and \colorB{nesting checks}}
\label{fig:step2-stm}
\end{figure}

\begin{figure}[p]
\begin{adjustbox}{frame=0.5pt 1em,max width=\textwidth,center,tabular=@{}c@{}}
\fbox{I/O transitions \qquad$P;\ \Theta\colorB{,\Sigma}\ \xrightarrow{a}\ Q;\ \Theta'\colorB{,\Sigma'}$}\\
\\
\begin{adjustbox}{max width=\linewidth,tabular=@{}>{$}r<{$} >{$}c<{$} >{$}l<{$} l@{}}
\ctxP{\term{putChar } c};\ \Theta\colorB{,\Sigma} & \xrightarrow{!c} & \ctxP{\term{return ()}};\ \Theta\colorB{,\Sigma} & \rul{PUTC} \\
\ctxP{\term{getChar}};\ \Theta\colorB{,\Sigma} & \xrightarrow{?c} & \ctxP{\term{return}\ c};\ \Theta\colorB{,\Sigma} & \rul{GETC} \\
\ctxP{\term{forkIO } M};\ \Theta\colorB{,\Sigma} & \rightarrow & (\ctxP{\term{return}\ t}\ \vert\ M_t);\ \Theta\colorB{,\Sigma} \quad t \notin \mathbb{P},\Theta\colorB{,\Sigma},M & \rul{FORK} \\
\ctxP{\term{catch}\ (\term{return}\ M)\ N};\ \Theta\colorB{,\Sigma} & \rightarrow & \ctxP{\term{return}\ M};\ \Theta\colorB{,\Sigma} & \rul{CATCH1} \\
\ctxP{\term{catch}\ (\term{throw}\ P)\ N};\ \Theta\colorB{,\Sigma} & \rightarrow & \ctxP{N\ P};\ \Theta\colorB{,\Sigma} & \rul{CATCH2} \\
\end{adjustbox}\\
\\
\inference{M\ \rightarrow\ N}{\ctxP{M};\ \Theta\colorB{,\Sigma}\ \rightarrow\ \ctxP{N};\ \Theta\colorB{,\Sigma}}[\rul{ADMIN}]\\
\\
\begin{adjustbox}{max width=\linewidth,tabular=@{}c@{}}
\inference{M;\ \Theta,\{\}\colorB{,\{\}} \xRightarrow{*}\ \term{return}\ N;\ \Theta',\Delta'\colorB{,\Sigma'}\\ \colorA{F\ N;\ (\Theta \cup \Delta')\colorB{,\{\}}\ \xrightarrow{*}\ \term{return}\ P;\ \hat{\Theta}\colorB{,\hat{\Sigma}}} & \colorB{(\Sigma'_{\textsc{rw}}\cap\hat{\Sigma}_{\textsc{w}})=\emptyset}}{\ctxP{\colorA{\term{atomicallyWithIO}}\ M\ \colorA{F}};\ \Theta\colorB{,\Sigma}\ \rightarrow\ \ctxP{\term{return}\ \colorA{P}};\ \colorA{(}\Theta'\colorA{{}\cup\hat{\Theta}{})}\colorB{,(\Sigma\cup\Sigma'\cup\hat{\Sigma})}}[\rul{ARET}]
\end{adjustbox}\\
\\
\begin{adjustbox}{max width=\linewidth,tabular=@{}c@{}}
\inference{M;\ \Theta,\{\}\colorB{,\{\}}\ \xRightarrow{*}\ \term{throw}\ N;\ \Theta',\Delta'\colorB{,\Sigma'}}{\ctxP{\colorA{\term{atomicallyWithIO}}\ M\ \colorA{F}};\ \Theta\colorB{,\Sigma}\ \rightarrow\ \ctxP{\term{throw}\ N};\ (\Theta \cup \Delta')\colorB{,(\Sigma\cup\Sigma'\vert_{\Delta'})}}[\rul{ATHROW\colorA{1}}]
\end{adjustbox}\\
\\
\begin{adjustbox}{max width=\linewidth,tabular=@{}c@{}}
\colorA{\inference{M;\ \Theta,\{\}\colorB{,\{\}}\ \xRightarrow{*}\ \term{return}\ N;\ \Theta',\Delta'\colorB{,\Sigma'}\\ F\ N;\ (\Theta \cup \Delta')\colorB{,\{\}}\ \xrightarrow{*}\ \term{throw}\ P;\ \hat{\Theta}\colorB{,\hat{\Sigma}} & \colorB{(\Sigma'_{\textsc{rw}}\cap\hat{\Sigma}_{\textsc{w}})=\emptyset}}{\ctxP{\term{atomicallyWithIO}\ M\ F};\ \Theta\colorB{,\Sigma}\ \rightarrow\ \ctxP{\term{throw}\ P};\ \hat{\Theta}\colorB{,(\Sigma\cup\Sigma'\vert_{\Delta'}\cup\hat{\Sigma})}}[\rul{ATHROW2}]}
\end{adjustbox}
\end{adjustbox}
\caption{I/O transitions of STM Haskell with \colorA{finalizers} and \colorB{nesting checks}}
\label{fig:step2-io}
\end{figure}

\clearpage

%------------------------------------------------------------

\subsection{Adding invariants}

GHC's STM implementation includes the \term{alwaysSucceeds} primitive, used to introduce dynamically-checked data invariants.
It was first described in \textcite{harris-peytonjones-2006} (where it was called \term{check}).
I have ignored invariants thus far since they increase the overall complexity of the semantics and are mostly orthogonal to finalizers.
It is now time to introduce them.
\Cref{fig:final-syntax,fig:final-stm,fig:final-io} constitute the complete and final syntax and semantics of STM Haskell with finalizers.

% sigma
Firstly, we add the set of invariants $\Omega$ as another piece of state associated with all I/O and STM transitions.
It is usually treated like $\Sigma$ or the heap.

% check1-2
Secondly, the new STM transition rules \rul{CHECK1} and \rul{CHECK2} describe the \term{alwaysSucceeds} primitive.
If the invariant holds at the point it is proposed, \rul{CHECK1} adds it to $\Omega$ and discards its evaluation effects.
\rul{CHECK2} applies when the invariant throws an exception.
As usual, the exception is propagated, the invariant discarded and only allocation effects are retained.
%TODO reference ghc bug?

% aret1, athrow2
Finally, the biggest changes happen on the I/O level:
\rul{ARET} becomes \rul{ARET1};
along with \rul{ATHROW2} it has a new premise that evaluates the invariants in place at the end of the STM transition.
When evaluating each invariant, its return value and all its heap effects are discarded.
Only if all invariants evaluate to \term{return} terms can we run the finalizer of \term{atomicallyWithIO}.
Note that it is required that the write set of the I/O transition ($\hat{\Sigma}_\textsc{w}$) is distinct from all of the memory locations referenced by the invariants ($\bigcup\Sigma'_i$).

% aret2
If any of the invariants throws an exception, the new rule \rul{ARET2} applies.
Similar to \rul{ATHROW1}, the exception is propagated to the I/O level and only allocation effects are preserved (including the allocation effects $\Delta'_i$ of the broken invariant $M_i$).
The I/O action is not executed in this case.

\begin{figure}[p]
\begin{adjustbox}{frame=0.5pt 1em,max width=\textwidth,center,tabular=@{}r >{$}r<{$} >{$}c<{$} >{$}l<{$}@{}}
& x,y & \in & Variable \\
& r,t & \in & Name \\
& c & \in & Char \\
\\
Value & V &  ::=  & r \svert c \svert \term{\textbackslash}\ x \term{->}\ M\\
& & \vert & \term{return}\ M \svert M\ \term{>>=}\ N \\
& & \vert & \term{putChar}\ c \svert \term{getChar} \\
& & \vert & \term{throw}\ M \svert \term{catch}\ M\ N \\
& & \vert & \term{retry} \svert M\ \term{`orElse`}\ N \\
& & \vert & \term{newTVar} \\
& & \vert & \term{readTVar}\ r \svert \term{writeTVar}\ r\ M \\
& & \vert & \term{forkIO}\ M \\
& & \vert & \term{alwaysSucceeds}\ M \\
& & \vert & \colorA{\term{atomicallyWithIO}\ M\ X} \\
\\
Term & M,N & ::= & x \svert V \svert M\ N \svert \ldots \\
\\
Thread soup & P,Q & ::= & M_t \svert (P\ \vert\ Q) \\
Heap & \Theta & ::= & r \hookrightarrow M \\
Allocations & \Delta & ::= & r \hookrightarrow M \\
\colorB{Touched references} & \colorB{\Sigma} & \colorB{::=} & \colorB{r \times (\textsc{r}\ \vert\ \textsc{w})} \\
Invariants & \Omega & ::= & \{M\} \\
\\
Evaluation contexts & \mathbb{S} & ::= & [\cdot] \svert \mathbb{S}\ \term{>>=}\ M \\
& \mathbb{P} & ::= & \mathbb{S}_t \svert (\mathbb{P}\ \vert\ P) \svert (P\ \vert\ \mathbb{P}) \\
Action & a & ::= & !c \svert ?c \svert \epsilon
\end{adjustbox}
\caption{The syntax of STM Haskell with invariants, \colorA{finalizers} and \colorB{nesting checks}}
\label{fig:final-syntax}
\end{figure}

\begin{figure}[p]
\begin{adjustbox}{frame=0.5pt 1em,max width=\textwidth,center,tabular=@{}c@{}}
\fbox{I/O transitions \qquad$P;\ \Theta,\colorB{\Sigma},\Omega\ \xrightarrow{a}\ Q;\ \Theta',\colorB{\Sigma'},\Omega'$}\\
\\
\begin{adjustbox}{max width=\linewidth,tabular=@{}>{$}r<{$} >{$}c<{$} >{$}l<{$} l@{}}
\ctxP{\term{putChar } c};\ \Theta\colorB{,\Sigma},\Omega & \xrightarrow{!c} &\ctxP{\term{return ()}};\ \Theta\colorB{,\Sigma},\Omega & \rul{PUTC} \\
\ctxP{\term{getChar}};\ \Theta\colorB{,\Sigma},\Omega & \xrightarrow{?c} & \ctxP{\term{return}\ c};\ \Theta\colorB{,\Sigma},\Omega & \rul{GETC} \\
\ctxP{\term{forkIO } M};\ \Theta\colorB{,\Sigma},\Omega & \rightarrow & (\ctxP{\term{return}\ t}\ \vert\ M_t);\ \Theta\colorB{,\Sigma},\Omega \quad t \notin \mathbb{P},\Theta\colorB{,\Sigma},\Omega,M & \rul{FORK} \\
\ctxP{\term{catch}\ (\term{return}\ M)\ N};\ \Theta\colorB{,\Sigma},\Omega & \rightarrow & \ctxP{\term{return}\ M};\ \Theta\colorB{,\Sigma},\Omega & \rul{CATCH1} \\
\ctxP{\term{catch}\ (\term{throw}\ P)\ N};\ \Theta\colorB{,\Sigma},\Omega & \rightarrow & \ctxP{N\ P};\ \Theta\colorB{,\Sigma},\Omega & \rul{CATCH2}
\end{adjustbox}\\
\\
\inference{M\ \rightarrow\ N}{\ctxP{M};\ \Theta\colorB{,\Sigma},\Omega\ \rightarrow\ \ctxP{N};\ \Theta\colorB{,\Sigma},\Omega}[\rul{ADMIN}]\\
\\
\begin{adjustbox}{max width=\linewidth,tabular=@{}c@{}}
\inference{M;\ \Theta\colorB{,\{\}},\{\},\Omega\ \xRightarrow{*}\ \term{return}\ N;\ \Theta',\Delta'\colorB{,\Sigma'},\Omega'\\ \forall M_i\in\Omega' : (M_i;\ \Theta',\{\}\colorB{,\{\}},\{\}\ \xRightarrow{*}\ \term{return}\ N_i;\ \Theta'_i,\Delta'_i\colorB{,\Sigma'_i},\Omega'_i)\\ \colorA{F\ N;\ (\Theta \cup \Delta'),\{\},\Omega\ \xrightarrow{*}\ \term{return}\ P;\ \hat{\Theta}\colorB{,\hat{\Sigma}},\hat{\Omega}} & \colorB{(\bigcup\Sigma'_i\cup\Sigma')_\textsc{rw} \cap\hat{\Sigma}_{\textsc{w}}=\emptyset}}{\ctxP{\colorA{\term{atomicallyWithIO}}\ M\ \colorA{F}};\ \Theta\colorB{,\Sigma},\Omega\ \rightarrow\ \ctxP{\term{return}\ \colorA{P}};\ \colorA{(}\Theta'\colorA{{}\cup{}\hat{\Theta})}\colorB{,(\Sigma\cup\Sigma'\cup\hat{\Sigma})},(\Omega'\cup\hat{\Omega})}[\rul{ARET1}]
\end{adjustbox}\\
\\
\begin{adjustbox}{max width=\linewidth,tabular=@{}c@{}}
\colorA{%
\inference{M;\ \Theta,\{\}\colorB{,\{\}},\Omega\ \xRightarrow{*}\ \term{return}\ N;\ \Theta',\Delta'\colorB{,\Sigma'},\Omega'\\ \forall M_i\in\Omega' : (M_i;\ \Theta',\{\}\colorB{,\{\}},\{\}\ \xRightarrow{*}\ \term{return}\ N_i;\ \Theta'_i,\Delta'_i\colorB{,\Sigma'_i},\Omega'_i)\\ F\ N;\ (\Theta \cup \Delta')\colorB{,\{\}},\Omega\ \xrightarrow{*}\ \term{throw}\ P;\ \hat{\Theta}\colorB{,\hat{\Sigma}},\hat{\Omega} & \colorB{(\bigcup\Sigma'_i\cup\Sigma')_\textsc{rw} \cap\hat{\Sigma}_{\textsc{w}}=\emptyset}}{\ctxP{\term{atomicallyWithIO}\ M\ F};\ \Theta\colorB{,\Sigma},\Omega\ \rightarrow\ \ctxP{\term{throw}\ P};\ \hat{\Theta}\colorB{,(\Sigma\cup\Sigma'\vert_{\Delta'}\cup\hat{\Sigma})},\hat{\Omega}}[\rul{ATHROW2}]}
\end{adjustbox}\\
\\
\begin{adjustbox}{max width=\linewidth,tabular=@{}c@{}}
\inference{M;\ \Theta,\{\}\colorB{,\{\}},\Omega\ \xRightarrow{*}\ \term{return}\ N;\ \Theta',\Delta'\colorB{,\Sigma'},\Omega'\\ \exists M_i\in\Omega' : (M_i;\ \Theta',\{\}\colorB{,\{\}},\{\}\ \xRightarrow{*}\ \term{throw}\ N_i;\ \Theta'_i,\Delta'_i\colorB{,\Sigma'_i},\Omega'_i)}{\ctxP{\colorA{\term{atomicallyWithIO}}\ M\ \colorA{F}};\ \Theta\colorB{,\Sigma},\Omega\ \rightarrow\  \ctxP{\term{throw}\ N_i};\ (\Theta\cup\Delta'\cup\Delta'_i)\colorB{,(\Sigma\cup\Sigma'\vert_{\Delta'}\cup\Sigma'_i\vert_{\Delta'_i})},\Omega}[\rul{ARET2}]
\end{adjustbox}\\
\\
\begin{adjustbox}{max width=\linewidth,tabular=@{}c@{}}
\inference{M;\ \Theta,\{\}\colorB{,\{\}},\Omega\ \xRightarrow{*}\ \term{throw}\ N;\ \Theta',\Delta'\colorB{,\Sigma'},\Omega'}{\ctxP{\colorA{\term{atomicallyWithIO}}\ M\ \colorA{F}};\ \Theta\colorB{,\Sigma},\Omega\ \rightarrow\ \ctxP{\term{throw}\ N};\ (\Theta \cup \Delta')\colorB{,(\Sigma\cup\Sigma'\vert_{\Delta'})},\Omega}[\rul{ATHROW\colorA{1}}]
\end{adjustbox}
\end{adjustbox}
\caption{I/O transitions of STM Haskell with invariants, \colorA{finalizers} and \colorB{nesting checks}}
\label{fig:final-io}
\end{figure}

\begin{figure}[p]
\begin{adjustbox}{frame=0.5pt 1em,max width=\textwidth,center,tabular=@{}c@{}}
\fbox{STM transitions \qquad $M;\ \Theta,\Delta\colorB{,\Sigma},\Omega\ \Rightarrow\ N;\ \Theta',\Delta'\colorB{,\Sigma'},\Omega'$}\\
\\
\begin{adjustbox}{max width=\linewidth,tabular=@{}>{$}r<{$} >{$}c<{$} >{$}l<{$} >{$}l<{$} l}
\ctxS{\term{readTVar}\ r};\ \Theta,\Delta\colorB{,\Sigma},\Omega & \Rightarrow & \ctxS{\term{return}\ \Theta(r)};\ \Theta,\Delta\colorB{,(\Sigma\cup\{(r,\textsc{r})\})},\Omega & \text{if } r \in dom(\Theta) & \rul{READ} \\
\ctxS{\term{writeTVar}\ r\ M};\ \Theta,\Delta\colorB{,\Sigma},\Omega & \Rightarrow & \ctxS{\term{return ()}};\ \Theta[r\mapsto M],\Delta\colorB{,(\Sigma\cup\{(r,\textsc{w})\})},\Omega & \text{if } r \in dom(\Theta) & \rul{WRITE} \\
\ctxS{\term{newTVar}\ M};\ \Theta,\Delta\colorB{,\Sigma},\Omega & \Rightarrow & \ctxS{\term{return}\ r};\ \Theta[r\mapsto M],\Delta[r\mapsto M]\colorB{,(\Sigma\cup\{(r,\textsc{w})\})},\Omega & r \notin dom(\Theta) & \rul{NEW}
\end{adjustbox}\\
\\
\begin{adjustbox}{max width=\linewidth,tabular=@{}c@{}}
\inference{M \quad \rightarrow \quad N}{\ctxS{M};\ \Theta,\Delta\colorB{,\Sigma},\Omega\  \Rightarrow\ \ctxS{N};\ \Theta,\Delta\colorB{,\Sigma},\Omega}[\rul{AADMIN}]
\end{adjustbox}\\
\\
\begin{adjustbox}{max width=\linewidth,tabular=@{}c@{}}
\inference{M_1;\ \Theta,\Delta\colorB{,\Sigma},\Omega\ \xRightarrow{*}\ \term{return}\ N;\ \Theta', \Delta'\colorB{,\Sigma'},\Omega'}{\ctxS{M_1\ \term{`orElse`}\ M_2};\ \Theta,\Delta\colorB{,\Sigma},\Omega\ \Rightarrow\  \ctxS{\term{return}\ N};\ \Theta',\Delta'\colorB{,\Sigma'},\Omega'}[\rul{OR1}]
\end{adjustbox}\\
\\
\begin{adjustbox}{max width=\linewidth,tabular=@{}c@{}}
\inference{M_1;\ \Theta,\Delta\colorB{,\Sigma},\Omega\ \xRightarrow{*}\ \term{throw}\ N;\ \Theta',\Delta'\colorB{,\Sigma'},\Omega'}{\ctxS{M_1\ \term{`orElse`}\ M_2};\ \Theta,\Delta\colorB{,\Sigma},\Omega\ \Rightarrow\ \ctxS{\term{throw}\ N};\ \Theta',\Delta'\colorB{,\Sigma'},\Omega'}[\rul{OR2}]
\end{adjustbox}\\
\\
\begin{adjustbox}{max width=\linewidth,tabular=@{}c@{}}
\inference{M_1;\ \Theta,\Delta\colorB{,\Sigma},\Omega\ \xRightarrow{*}\ \term{retry} ;\ \Theta',\Delta'\colorB{,\Sigma'},\Omega'}{\ctxS{M_1\ \term{`orElse`}\ M_2};\ \Theta,\Delta\colorB{,\Sigma},\Omega\ \Rightarrow\ \ctxS{M_2};\ \Theta,\Delta\colorB{,\Sigma},\Omega}[\rul{OR3}]
\end{adjustbox}\\
\\
\begin{adjustbox}{max width=\linewidth,tabular=@{}c@{}}
\inference{M;\ \Theta,\{\}\colorB{,\Sigma},\Omega\ \xRightarrow{*}\ \term{return}\ P;\ \Theta',\Delta'\colorB{,\Sigma'},\Omega'}{\ctxS{\term{catch}\ M\ N};\ \Theta,\Delta\colorB{,\Sigma},\Omega\ \Rightarrow\ \ctxS{\term{return}\ P};\ \Theta',(\Delta\cup\Delta')\colorB{,\Sigma'},\Omega'}[\rul{XSTM1}]
\end{adjustbox}\\
\\
\begin{adjustbox}{max width=\linewidth,tabular=@{}c@{}}
\inference{M;\ \Theta,\{\}\colorB{,\Sigma},\Omega\ \xRightarrow{*}\ \term{throw}\ P;\ \Theta',\Delta'\colorB{,\Sigma'},\Omega'}{\ctxS{\term{catch}\ M\ N};\ \Theta,\Delta\colorB{,\Sigma},\Omega\ \Rightarrow\ \ctxS{N\ P};\ (\Theta\cup\Delta'),(\Delta\cup\Delta')\colorB{,(\Sigma\cup\Sigma'\vert_{\Delta'})},\Omega}[\rul{XSTM2}]
\end{adjustbox}\\
\\
\begin{adjustbox}{max width=\linewidth,tabular=@{}c@{}}
\inference{M;\ \Theta,\{\}\colorB{,\Sigma},\Omega\ \xRightarrow{*}\ \term{retry};\ \Theta',\Delta'\colorB{,\Sigma'},\Omega'}{\ctxS{\term{catch}\ M\ N};\ \Theta,\Delta\colorB{,\Sigma},\Omega\ \Rightarrow\ \ctxS{\term{retry}};\ \Theta,\Delta\colorB{,\Sigma},\Omega}[\rul{XSTM3}]
\end{adjustbox}\\
\\
\begin{adjustbox}{max width=\linewidth,tabular=@{}c@{}}
\inference{M;\ \Theta,\{\}\colorB{,\Sigma},\Omega\ \xRightarrow{*}\ \term{return}\ N;\ \Theta',\Delta'\colorB{,\Sigma'},\Omega'}{\ctxS{\term{alwaysSucceeds}\ M};\ \Theta,\Delta\colorB{,\Sigma},\Omega\ \Rightarrow\ \ctxS{\term{return}\ \texttt{()}};\ \Theta,\Delta\colorB{,\Sigma},(\Omega\cup\{M\})}[\rul{CHECK1}]
\end{adjustbox}\\
\\
\begin{adjustbox}{max width=\linewidth,tabular=@{}c@{}}
\inference{M;\ \Theta,\{\}\colorB{,\Sigma},\Omega\ \xRightarrow{*}\ \term{throw}\ N;\ \Theta',\Delta'\colorB{,\Sigma'},\Omega'}{\ctxS{\term{alwaysSucceeds}\ M};\ \Theta,\Delta\colorB{,\Sigma},\Omega\ \Rightarrow\ \ctxS{\term{throw}\ N};\ (\Theta\cup\Delta'),(\Delta\cup\Delta')\colorB{,(\Sigma\cup\Sigma'\vert_{\Delta'})},\Omega}[\rul{CHECK2}]
\end{adjustbox}
\end{adjustbox}
\caption{STM transitions of STM Haskell with invariants, \colorA{finalizers} and \colorB{nesting checks}}
\label{fig:final-stm}
\end{figure}

\clearpage